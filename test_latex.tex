\documentclass[a4paper,10pt]{article}

% Essential packages
\usepackage{fontspec}
\usepackage{xcolor}
\usepackage{tikz}
\usepackage{multicol}
\usepackage{lmodern}

% Set margins
\usepackage[margin=1cm]{geometry}

% Define colors
\definecolor{highlight}{RGB}{227, 76, 60}

% Font declarations for Indian scripts
\newfontfamily\LatinFont[Path=./fonts/]{texgyrepagella-regular.otf}
\newfontfamily\DevanagariFont[Path=./fonts/]{NotoSansDevanagari-Regular.ttf}
\newfontfamily\BengaliFont[Path=./fonts/]{NotoSansBengali-Regular.ttf}
\newfontfamily\GujaratiFont[Path=./fonts/]{NotoSansGujarati-Regular.ttf}
\newfontfamily\GurmukhiFont[Path=./fonts/]{NotoSansGurmukhi-Regular.ttf}
\newfontfamily\KannadaFont[Path=./fonts/]{NotoSansKannada-Regular.ttf}
\newfontfamily\MalayalamFont[Path=./fonts/]{NotoSansMalayalam-Regular.ttf}
\newfontfamily\TamilFont[Path=./fonts/]{NotoSansTamil-Regular.ttf}
\newfontfamily\TeluguFont[Path=./fonts/]{NotoSansTelugu-Regular.ttf}
\newfontfamily\OdiaFont[Path=./fonts/]{NotoSansOriya-Regular.ttf}
\newfontfamily\UrduFont[Path=./fonts/]{NotoSansArabic-Regular.ttf}
\newfontfamily\AssameseFont[Path=./fonts/]{NotoSansBengali-Regular.ttf}
\newfontfamily\KashmiriFont[Path=./fonts/]{NotoSansArabic-Regular.ttf}
\newfontfamily\SindhiFont[Path=./fonts/]{NotoSansArabic-Regular.ttf}
\newfontfamily\ManipuriFont[Path=./fonts/]{NotoSansMeeteiMayek-Regular.ttf}
\newfontfamily\OlChikiFont[Path=./fonts/]{NotoSansOlChiki-Regular.ttf}

\begin{document}

% Title
\begin{center}
\Huge\textbf{LaTeX Test}
\end{center}

% Grid setup
\begin{center}
\begin{tikzpicture}[inner sep=0pt]
\draw[line width=0.5pt] (0,0) rectangle (2.0,2.0);  % Add 0.5cm margin
\node at (0.5,1.5) {\normalsize\LatinFont {3}};
\node at (1.0,1.5) {\normalsize\DevanagariFont {2}};
\node at (1.5,1.5) {\normalsize\BengaliFont {3}};
\node at (0.5,1.0) {\normalsize\GujaratiFont {4}};
\node at (1.0,1.0) {\normalsize\GurmukhiFont \textcolor{highlight}{5}};
\node at (1.5,1.0) {\normalsize\KannadaFont {6}};
\node at (0.5,0.5) {\normalsize\MalayalamFont {7}};
\node at (1.0,0.5) {\normalsize\TamilFont {8}};
\node at (1.5,0.5) {\normalsize\TeluguFont {9}};
\end{tikzpicture}
\end{center}

\vspace{0.5cm}
\begin{center}
\textbf{Statistics}
\end{center}

\begin{multicols}{2}
\begin{itemize}
\item Number of digits: 9
\item Grid dimensions: 3 $\times$ 3
\item Indian scripts used: 16
\item Random seed: 42
\end{itemize}

\columnbreak

\textbf{Scripts Usage:}
\begin{itemize}
\item Latin: 1
\item Devanagari: 1
\item Bengali: 1
\item Gujarati: 1
\item Gurmukhi: 1
\item Kannada: 1
\item Malayalam: 1
\item Tamil: 1
\item Telugu: 1
\end{itemize}
\end{multicols}

\vspace{1cm}
\begin{center}
\small This visualization was generated using a Python program. The 3 in Latin stands as the first digit of $\pi$.\\
Red digits form the shape of the $\pi$ symbol within the grid.
\end{center}

\end{document}
